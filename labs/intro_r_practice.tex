% Options for packages loaded elsewhere
\PassOptionsToPackage{unicode}{hyperref}
\PassOptionsToPackage{hyphens}{url}
%
\documentclass[
]{article}
\usepackage{lmodern}
\usepackage{amssymb,amsmath}
\usepackage{ifxetex,ifluatex}
\ifnum 0\ifxetex 1\fi\ifluatex 1\fi=0 % if pdftex
  \usepackage[T1]{fontenc}
  \usepackage[utf8]{inputenc}
  \usepackage{textcomp} % provide euro and other symbols
\else % if luatex or xetex
  \usepackage{unicode-math}
  \defaultfontfeatures{Scale=MatchLowercase}
  \defaultfontfeatures[\rmfamily]{Ligatures=TeX,Scale=1}
\fi
% Use upquote if available, for straight quotes in verbatim environments
\IfFileExists{upquote.sty}{\usepackage{upquote}}{}
\IfFileExists{microtype.sty}{% use microtype if available
  \usepackage[]{microtype}
  \UseMicrotypeSet[protrusion]{basicmath} % disable protrusion for tt fonts
}{}
\makeatletter
\@ifundefined{KOMAClassName}{% if non-KOMA class
  \IfFileExists{parskip.sty}{%
    \usepackage{parskip}
  }{% else
    \setlength{\parindent}{0pt}
    \setlength{\parskip}{6pt plus 2pt minus 1pt}}
}{% if KOMA class
  \KOMAoptions{parskip=half}}
\makeatother
\usepackage{xcolor}
\IfFileExists{xurl.sty}{\usepackage{xurl}}{} % add URL line breaks if available
\IfFileExists{bookmark.sty}{\usepackage{bookmark}}{\usepackage{hyperref}}
\hypersetup{
  pdftitle={IMT 573: Lab 1 - Introduction to R and RStudio Server},
  pdfauthor={Alex Davis},
  hidelinks,
  pdfcreator={LaTeX via pandoc}}
\urlstyle{same} % disable monospaced font for URLs
\usepackage[margin=1in]{geometry}
\usepackage{color}
\usepackage{fancyvrb}
\newcommand{\VerbBar}{|}
\newcommand{\VERB}{\Verb[commandchars=\\\{\}]}
\DefineVerbatimEnvironment{Highlighting}{Verbatim}{commandchars=\\\{\}}
% Add ',fontsize=\small' for more characters per line
\usepackage{framed}
\definecolor{shadecolor}{RGB}{248,248,248}
\newenvironment{Shaded}{\begin{snugshade}}{\end{snugshade}}
\newcommand{\AlertTok}[1]{\textcolor[rgb]{0.94,0.16,0.16}{#1}}
\newcommand{\AnnotationTok}[1]{\textcolor[rgb]{0.56,0.35,0.01}{\textbf{\textit{#1}}}}
\newcommand{\AttributeTok}[1]{\textcolor[rgb]{0.77,0.63,0.00}{#1}}
\newcommand{\BaseNTok}[1]{\textcolor[rgb]{0.00,0.00,0.81}{#1}}
\newcommand{\BuiltInTok}[1]{#1}
\newcommand{\CharTok}[1]{\textcolor[rgb]{0.31,0.60,0.02}{#1}}
\newcommand{\CommentTok}[1]{\textcolor[rgb]{0.56,0.35,0.01}{\textit{#1}}}
\newcommand{\CommentVarTok}[1]{\textcolor[rgb]{0.56,0.35,0.01}{\textbf{\textit{#1}}}}
\newcommand{\ConstantTok}[1]{\textcolor[rgb]{0.00,0.00,0.00}{#1}}
\newcommand{\ControlFlowTok}[1]{\textcolor[rgb]{0.13,0.29,0.53}{\textbf{#1}}}
\newcommand{\DataTypeTok}[1]{\textcolor[rgb]{0.13,0.29,0.53}{#1}}
\newcommand{\DecValTok}[1]{\textcolor[rgb]{0.00,0.00,0.81}{#1}}
\newcommand{\DocumentationTok}[1]{\textcolor[rgb]{0.56,0.35,0.01}{\textbf{\textit{#1}}}}
\newcommand{\ErrorTok}[1]{\textcolor[rgb]{0.64,0.00,0.00}{\textbf{#1}}}
\newcommand{\ExtensionTok}[1]{#1}
\newcommand{\FloatTok}[1]{\textcolor[rgb]{0.00,0.00,0.81}{#1}}
\newcommand{\FunctionTok}[1]{\textcolor[rgb]{0.00,0.00,0.00}{#1}}
\newcommand{\ImportTok}[1]{#1}
\newcommand{\InformationTok}[1]{\textcolor[rgb]{0.56,0.35,0.01}{\textbf{\textit{#1}}}}
\newcommand{\KeywordTok}[1]{\textcolor[rgb]{0.13,0.29,0.53}{\textbf{#1}}}
\newcommand{\NormalTok}[1]{#1}
\newcommand{\OperatorTok}[1]{\textcolor[rgb]{0.81,0.36,0.00}{\textbf{#1}}}
\newcommand{\OtherTok}[1]{\textcolor[rgb]{0.56,0.35,0.01}{#1}}
\newcommand{\PreprocessorTok}[1]{\textcolor[rgb]{0.56,0.35,0.01}{\textit{#1}}}
\newcommand{\RegionMarkerTok}[1]{#1}
\newcommand{\SpecialCharTok}[1]{\textcolor[rgb]{0.00,0.00,0.00}{#1}}
\newcommand{\SpecialStringTok}[1]{\textcolor[rgb]{0.31,0.60,0.02}{#1}}
\newcommand{\StringTok}[1]{\textcolor[rgb]{0.31,0.60,0.02}{#1}}
\newcommand{\VariableTok}[1]{\textcolor[rgb]{0.00,0.00,0.00}{#1}}
\newcommand{\VerbatimStringTok}[1]{\textcolor[rgb]{0.31,0.60,0.02}{#1}}
\newcommand{\WarningTok}[1]{\textcolor[rgb]{0.56,0.35,0.01}{\textbf{\textit{#1}}}}
\usepackage{graphicx,grffile}
\makeatletter
\def\maxwidth{\ifdim\Gin@nat@width>\linewidth\linewidth\else\Gin@nat@width\fi}
\def\maxheight{\ifdim\Gin@nat@height>\textheight\textheight\else\Gin@nat@height\fi}
\makeatother
% Scale images if necessary, so that they will not overflow the page
% margins by default, and it is still possible to overwrite the defaults
% using explicit options in \includegraphics[width, height, ...]{}
\setkeys{Gin}{width=\maxwidth,height=\maxheight,keepaspectratio}
% Set default figure placement to htbp
\makeatletter
\def\fps@figure{htbp}
\makeatother
\setlength{\emergencystretch}{3em} % prevent overfull lines
\providecommand{\tightlist}{%
  \setlength{\itemsep}{0pt}\setlength{\parskip}{0pt}}
\setcounter{secnumdepth}{-\maxdimen} % remove section numbering

\title{IMT 573: Lab 1 - Introduction to R and RStudio Server}
\author{Alex Davis}
\date{October 06 2020}

\begin{document}
\maketitle

\hypertarget{objectives}{%
\subsubsection{Objectives}\label{objectives}}

In this demo we will get a first look at writing R code for data
science. We will review basic R syntax and take a look at the different
R data structures we will use throughout the course. We will learn how
to find and access existing datasets, and even make our first
visualization of data!

This demonstration also give you a glimpse of writing reproducible data
science reports using Rmarkdown. We will talk more about this next week!

\hypertarget{a-little-background-on-r}{%
\subsubsection{A little background on
R}\label{a-little-background-on-r}}

Everything in \textbf{R} is an object - data, functions, everything!
When you type in commands what happens:

\begin{itemize}
\tightlist
\item
  \textbf{R} tries to interpret what you've asked it to do (evaluation)
\item
  If it understands what you've told it to do, no problem
\item
  If it does not understand, it will likely give you an error or warning
  message
\end{itemize}

Some commands trigger \textbf{R} to print something to the screen,
others don't.

If you type in an incomplete command, \textbf{R} will usually respond by
changing the command prompt to the \(+\) character to demonstrate it is
waiting for something more. A \(>\) indicated the beginning of a line.
You shouldn't have to consider this too much because you should always
write your code in a \emph{script} rather than the Console.

\hypertarget{getting-started}{%
\subsubsection{Getting started}\label{getting-started}}

Welcome to RStudio! RStudio is an integrated development environment
(IDE) for R. It comes with a lot of nifty functionality to make it
easier for us to do data science! Take a tour of RStudio using the
online learning center or just play around with it after class today.

First, let's consider setting up our environment. In this report, we
will be able to write text, as we have done already, but we will also be
able to write code! Code should be contained in a \emph{code chunk}.
Code chunks are marked as follows:

\begin{Shaded}
\begin{Highlighting}[]
\CommentTok{# Hello R!}
\CommentTok{# This is a code comment.}
\CommentTok{# Code comments help you document your coding proces!}

\CommentTok{#This is a code chunk named "our first code chunk", which is the name of the code chunk;}
\CommentTok{# Including code chunk name is optional, but this practice will help you create well documented code}
\end{Highlighting}
\end{Shaded}

\hypertarget{introduction-to-basic-r-syntax}{%
\subsubsection{Introduction to basic R
syntax}\label{introduction-to-basic-r-syntax}}

Let's take a look at some basic R syntax. Remember everything in R is an
object! We also want to follow the \texttt{tidyverse} style guide for
writing code. Variable and function names should use only lowercase
letters, numbers, and \_.

\begin{Shaded}
\begin{Highlighting}[]
\DecValTok{1} \OperatorTok{+}\StringTok{ }\DecValTok{3}       \CommentTok{# evaluation}
\end{Highlighting}
\end{Shaded}

\begin{verbatim}
## [1] 4
\end{verbatim}

\begin{Shaded}
\begin{Highlighting}[]
\NormalTok{a <-}\StringTok{ }\DecValTok{3}  \CommentTok{# assignment. <- is the assignment symbol}
\NormalTok{a               }\CommentTok{# evaluation}
\end{Highlighting}
\end{Shaded}

\begin{verbatim}
## [1] 3
\end{verbatim}

\begin{Shaded}
\begin{Highlighting}[]
\NormalTok{a<-}\DecValTok{3}        \CommentTok{# spacing does not matter}
\NormalTok{a    <-}\StringTok{    }\DecValTok{3}        \CommentTok{# spacing does not matter}

\KeywordTok{sqrt}\NormalTok{(a)         }\CommentTok{# use the square root function}
\end{Highlighting}
\end{Shaded}

\begin{verbatim}
## [1] 1.732051
\end{verbatim}

\begin{Shaded}
\begin{Highlighting}[]
\NormalTok{b <-}\StringTok{ }\KeywordTok{sqrt}\NormalTok{(a)        }\CommentTok{# use function and save result}
\NormalTok{b}
\end{Highlighting}
\end{Shaded}

\begin{verbatim}
## [1] 1.732051
\end{verbatim}

\begin{Shaded}
\begin{Highlighting}[]
\CommentTok{# x             # evaluate something that is not there}

\NormalTok{a }\OperatorTok{==}\StringTok{ }\NormalTok{b          }\CommentTok{# is a equal to b?         }
\end{Highlighting}
\end{Shaded}

\begin{verbatim}
## [1] FALSE
\end{verbatim}

\begin{Shaded}
\begin{Highlighting}[]
\NormalTok{a }\OperatorTok{!=}\StringTok{ }\NormalTok{b          }\CommentTok{# is a not equal to b?}
\end{Highlighting}
\end{Shaded}

\begin{verbatim}
## [1] TRUE
\end{verbatim}

\begin{Shaded}
\begin{Highlighting}[]
\CommentTok{# list objects in the R environment}
\CommentTok{# (same as viewing the "Workspace" pane in RStudio)}
\KeywordTok{ls}\NormalTok{()}
\end{Highlighting}
\end{Shaded}

\begin{verbatim}
## [1] "a" "b"
\end{verbatim}

\begin{Shaded}
\begin{Highlighting}[]
\KeywordTok{rm}\NormalTok{(a)           }\CommentTok{# remove a single object}
\KeywordTok{rm}\NormalTok{(}\DataTypeTok{list=}\KeywordTok{ls}\NormalTok{())       }\CommentTok{# remove everything from the environment}
\end{Highlighting}
\end{Shaded}

\hypertarget{getting-help-in-r}{%
\subsubsection{Getting help in R}\label{getting-help-in-r}}

\begin{Shaded}
\begin{Highlighting}[]
\CommentTok{# get help with R generally}
\CommentTok{# (same as viewing the "Help" pane in RStudio)}
\KeywordTok{help.start}\NormalTok{()}
\end{Highlighting}
\end{Shaded}

\begin{verbatim}
## starting httpd help server ... done
\end{verbatim}

\begin{verbatim}
## If the browser launched by '/usr/bin/open' is already running, it is
##     *not* restarted, and you must switch to its window.
## Otherwise, be patient ...
\end{verbatim}

\begin{Shaded}
\begin{Highlighting}[]
\CommentTok{# More targeted help}
\NormalTok{?sqrt           }\CommentTok{# get specific help for a function}

\KeywordTok{apropos}\NormalTok{(}\StringTok{"sq"}\NormalTok{)       }\CommentTok{# regular expression match. What do you do when you can't really recall the exact function name}
\end{Highlighting}
\end{Shaded}

\begin{verbatim}
## [1] "chisq.test" "dchisq"     "pchisq"     "qchisq"     "rchisq"    
## [6] "sqrt"       "sQuote"
\end{verbatim}

\hypertarget{data-types-in-r}{%
\subsubsection{Data Types in R}\label{data-types-in-r}}

There are numerous data types in R that store various kinds of data. The
four main types of data most likely to be used are numeric, character
(string), Date (time-based) and logical (TRUE/FALSE).

\begin{Shaded}
\begin{Highlighting}[]
\CommentTok{# Check the type of data contained in a variable with the class function. }
\NormalTok{x <-}\StringTok{ }\DecValTok{3}
\NormalTok{x}
\end{Highlighting}
\end{Shaded}

\begin{verbatim}
## [1] 3
\end{verbatim}

\begin{Shaded}
\begin{Highlighting}[]
\KeywordTok{class}\NormalTok{(x)}
\end{Highlighting}
\end{Shaded}

\begin{verbatim}
## [1] "numeric"
\end{verbatim}

\begin{Shaded}
\begin{Highlighting}[]
\CommentTok{# Numeric data type -- Testing whether a variable is numeric}
\NormalTok{a <-}\StringTok{ }\DecValTok{45}
\NormalTok{a}
\end{Highlighting}
\end{Shaded}

\begin{verbatim}
## [1] 45
\end{verbatim}

\begin{Shaded}
\begin{Highlighting}[]
\KeywordTok{class}\NormalTok{(a)}
\end{Highlighting}
\end{Shaded}

\begin{verbatim}
## [1] "numeric"
\end{verbatim}

\begin{Shaded}
\begin{Highlighting}[]
\CommentTok{# Character data}
\NormalTok{b <-}\StringTok{ "b"}
\NormalTok{b}
\end{Highlighting}
\end{Shaded}

\begin{verbatim}
## [1] "b"
\end{verbatim}

\begin{Shaded}
\begin{Highlighting}[]
\KeywordTok{class}\NormalTok{(b)}
\end{Highlighting}
\end{Shaded}

\begin{verbatim}
## [1] "character"
\end{verbatim}

\begin{Shaded}
\begin{Highlighting}[]
\CommentTok{# Date type -- Dealing with dates and times can be difficult in any language, and to further complicate matters R has numerous different types of dates. }
\NormalTok{c <-}\StringTok{ }\KeywordTok{Sys.Date}\NormalTok{()}
\NormalTok{c}
\end{Highlighting}
\end{Shaded}

\begin{verbatim}
## [1] "2020-10-06"
\end{verbatim}

\begin{Shaded}
\begin{Highlighting}[]
\KeywordTok{class}\NormalTok{(c)}
\end{Highlighting}
\end{Shaded}

\begin{verbatim}
## [1] "Date"
\end{verbatim}

\begin{Shaded}
\begin{Highlighting}[]
\CommentTok{# Logical data type: True/False}
\NormalTok{d <-}\StringTok{ }\OtherTok{FALSE}
\NormalTok{d}
\end{Highlighting}
\end{Shaded}

\begin{verbatim}
## [1] FALSE
\end{verbatim}

\begin{Shaded}
\begin{Highlighting}[]
\KeywordTok{class}\NormalTok{(d)}
\end{Highlighting}
\end{Shaded}

\begin{verbatim}
## [1] "logical"
\end{verbatim}

\begin{Shaded}
\begin{Highlighting}[]
\CommentTok{# Factor vectors: Ideal for representing categorical variables (More on that later)}
\NormalTok{vect <-}\StringTok{ }\KeywordTok{c}\NormalTok{(}\DecValTok{1}\OperatorTok{:}\DecValTok{10}\NormalTok{)}
\KeywordTok{class}\NormalTok{(vect)}
\end{Highlighting}
\end{Shaded}

\begin{verbatim}
## [1] "integer"
\end{verbatim}

\hypertarget{vectors-and-matrices-in-r}{%
\subsubsection{Vectors and matrices in
R}\label{vectors-and-matrices-in-r}}

\hypertarget{vectors-in-r-a-collection-of-elements-all-of-the-same-data-type}{%
\paragraph{Vectors in R: A collection of elements all of the same data
type}\label{vectors-in-r-a-collection-of-elements-all-of-the-same-data-type}}

\begin{Shaded}
\begin{Highlighting}[]
\CommentTok{# Creating vectors using c() function or the "combine" operator}
\NormalTok{a <-}\StringTok{ }\KeywordTok{c}\NormalTok{(}\DecValTok{1}\NormalTok{,}\DecValTok{3}\NormalTok{,}\DecValTok{5}\NormalTok{,}\DecValTok{7}\NormalTok{) }
\NormalTok{a}
\end{Highlighting}
\end{Shaded}

\begin{verbatim}
## [1] 1 3 5 7
\end{verbatim}

\begin{Shaded}
\begin{Highlighting}[]
\CommentTok{# select the second element}
\NormalTok{a[}\DecValTok{2}\NormalTok{]}
\end{Highlighting}
\end{Shaded}

\begin{verbatim}
## [1] 3
\end{verbatim}

\begin{Shaded}
\begin{Highlighting}[]
\CommentTok{# also works with strings. let's see how.}
\NormalTok{b <-}\StringTok{ }\KeywordTok{c}\NormalTok{(}\StringTok{"red"}\NormalTok{,}\StringTok{"green"}\NormalTok{,}\StringTok{"blue"}\NormalTok{,}\StringTok{"purple"}\NormalTok{)   }
\NormalTok{b}
\end{Highlighting}
\end{Shaded}

\begin{verbatim}
## [1] "red"    "green"  "blue"   "purple"
\end{verbatim}

\begin{Shaded}
\begin{Highlighting}[]
\CommentTok{# all colors except blue}
\NormalTok{b <-}\StringTok{ }\KeywordTok{c}\NormalTok{(}\StringTok{"red"}\NormalTok{,}\StringTok{"green"}\NormalTok{,}\StringTok{"blue"}\NormalTok{,}\StringTok{"purple"}\NormalTok{)   }
\NormalTok{b[}\OperatorTok{-}\DecValTok{3}\NormalTok{]}
\end{Highlighting}
\end{Shaded}

\begin{verbatim}
## [1] "red"    "green"  "purple"
\end{verbatim}

\begin{Shaded}
\begin{Highlighting}[]
\CommentTok{#all numbers less than 5}
\NormalTok{a <-}\StringTok{ }\KeywordTok{c}\NormalTok{(}\DecValTok{1}\NormalTok{,}\DecValTok{3}\NormalTok{,}\DecValTok{5}\NormalTok{,}\DecValTok{7}\NormalTok{) }
\NormalTok{a[}\OperatorTok{-}\DecValTok{3}\NormalTok{]}
\end{Highlighting}
\end{Shaded}

\begin{verbatim}
## [1] 1 3 7
\end{verbatim}

\begin{Shaded}
\begin{Highlighting}[]
\CommentTok{#add a new element}
\NormalTok{b[}\DecValTok{5}\NormalTok{] <-}\StringTok{ "yellow"}


\CommentTok{#change the first element}
\NormalTok{b[}\DecValTok{1}\NormalTok{] <-}\StringTok{ "gold"}


\CommentTok{# combine by applying recursively}
\NormalTok{a <-}\StringTok{ }\KeywordTok{c}\NormalTok{(a,a)}


\CommentTok{# mixing types---what happens?}
\NormalTok{c <-}\StringTok{ }\KeywordTok{c}\NormalTok{(a,b)}
\NormalTok{c}
\end{Highlighting}
\end{Shaded}

\begin{verbatim}
##  [1] "1"      "3"      "5"      "7"      "1"      "3"      "5"      "7"     
##  [9] "gold"   "green"  "blue"   "purple" "yellow"
\end{verbatim}

\begin{Shaded}
\begin{Highlighting}[]
\CommentTok{# Sequences and replication}

\CommentTok{#creating a vector using sequence. sequence from 1 to 5}
\NormalTok{a <-}\StringTok{ }\KeywordTok{seq}\NormalTok{(}\DataTypeTok{from=}\DecValTok{1}\NormalTok{,}\DataTypeTok{to=}\DecValTok{5}\NormalTok{,}\DataTypeTok{by=}\DecValTok{1}\NormalTok{)}
\NormalTok{b <-}\StringTok{ }\DecValTok{1}\OperatorTok{:}\DecValTok{5}                    \CommentTok{# a shortcut!}

\CommentTok{#creating a vector using sequence. sequence from 1 to 10, steps of 2}
\NormalTok{a <-}\StringTok{ }\KeywordTok{seq}\NormalTok{(}\DataTypeTok{from=}\DecValTok{1}\NormalTok{,}\DataTypeTok{to=}\DecValTok{10}\NormalTok{,}\DataTypeTok{by=}\DecValTok{2}\NormalTok{)}

\CommentTok{# replicate elements of a vector}
\KeywordTok{rep}\NormalTok{(}\DecValTok{1}\NormalTok{,}\DataTypeTok{times=}\DecValTok{3}\NormalTok{)}
\end{Highlighting}
\end{Shaded}

\begin{verbatim}
## [1] 1 1 1
\end{verbatim}

\begin{Shaded}
\begin{Highlighting}[]
\CommentTok{# Any, all, and which (with vectors)}
\KeywordTok{rep}\NormalTok{(}\DecValTok{1}\OperatorTok{:}\DecValTok{10}\NormalTok{,}\DataTypeTok{times=}\DecValTok{3}\NormalTok{)}
\end{Highlighting}
\end{Shaded}

\begin{verbatim}
##  [1]  1  2  3  4  5  6  7  8  9 10  1  2  3  4  5  6  7  8  9 10  1  2  3  4  5
## [26]  6  7  8  9 10
\end{verbatim}

\begin{Shaded}
\begin{Highlighting}[]
\CommentTok{# How long is the vector?}
\KeywordTok{length}\NormalTok{(a)}
\end{Highlighting}
\end{Shaded}

\begin{verbatim}
## [1] 5
\end{verbatim}

\hypertarget{element-wise-operations-on-vectors}{%
\subsubsection{Element-wise operations on
vectors}\label{element-wise-operations-on-vectors}}

\begin{Shaded}
\begin{Highlighting}[]
\CommentTok{# Most arithmetic operators are applied element-wise:}
\NormalTok{a <-}\StringTok{ }\DecValTok{1}\OperatorTok{:}\DecValTok{5}        \CommentTok{# create a vector}
\NormalTok{a }\OperatorTok{+}\StringTok{ }\DecValTok{1}       \CommentTok{# addition}
\end{Highlighting}
\end{Shaded}

\begin{verbatim}
## [1] 2 3 4 5 6
\end{verbatim}

\hypertarget{vectorized-functions.-transforming-vectors-by-applying-functions}{%
\subsubsection{vectorized functions. Transforming vectors by applying
functions}\label{vectorized-functions.-transforming-vectors-by-applying-functions}}

\begin{Shaded}
\begin{Highlighting}[]
\CommentTok{#vector of numbers}
\NormalTok{nums <-}\StringTok{ }\KeywordTok{c}\NormalTok{(}\FloatTok{3.98}\NormalTok{, }\FloatTok{8.2}\NormalTok{, }\FloatTok{10.5}\NormalTok{, }\FloatTok{3.6}\NormalTok{, }\FloatTok{5.5}\NormalTok{)}
\KeywordTok{round}\NormalTok{(nums, }\DecValTok{1}\NormalTok{) }\CommentTok{# round to nearest whole number or number of decimal places, if specified}
\end{Highlighting}
\end{Shaded}

\begin{verbatim}
## [1]  4.0  8.2 10.5  3.6  5.5
\end{verbatim}

\hypertarget{from-vectors-1d-collection-of-data-to-matrices-2d-collection-of-data}{%
\subsubsection{From vectors (1D collection of data) to matrices (2D
collection of
data)}\label{from-vectors-1d-collection-of-data-to-matrices-2d-collection-of-data}}

\begin{Shaded}
\begin{Highlighting}[]
\CommentTok{# create a matrix the "formal" way...}
\NormalTok{a <-}\StringTok{ }\KeywordTok{matrix}\NormalTok{(}\DecValTok{1}\OperatorTok{:}\DecValTok{25}\NormalTok{, }\DataTypeTok{nrow=}\DecValTok{5}\NormalTok{, }\DataTypeTok{ncol =} \DecValTok{5}\NormalTok{)}
\NormalTok{a}
\end{Highlighting}
\end{Shaded}

\begin{verbatim}
##      [,1] [,2] [,3] [,4] [,5]
## [1,]    1    6   11   16   21
## [2,]    2    7   12   17   22
## [3,]    3    8   13   18   23
## [4,]    4    9   14   19   24
## [5,]    5   10   15   20   25
\end{verbatim}

\hypertarget{element-wise-operations-on-matrices-same-principles-as-vectors}{%
\subsubsection{Element-wise operations on matrices (same principles as
vectors)}\label{element-wise-operations-on-matrices-same-principles-as-vectors}}

\hypertarget{factor-variable}{%
\subsubsection{Factor Variable}\label{factor-variable}}

Factors consist of a finite set of categories (primarily used for
categorical variables). Factors also optimize for space. Instead of
storing each of the character strings, example `small', `medium', it
will store a number and R will remember the relationship between the
label and the string. Example: 1 for `small', 2 for `medium', etc. Let's
see with an example

\begin{Shaded}
\begin{Highlighting}[]
\CommentTok{# A character vector of shirt sizes}
\NormalTok{shirt_sizes <-}\StringTok{ }\KeywordTok{c}\NormalTok{(}\StringTok{'small'}\NormalTok{, }\StringTok{'medium'}\NormalTok{, }\StringTok{'small'}\NormalTok{, }\StringTok{'large'}\NormalTok{, }\StringTok{'medium'}\NormalTok{, }\StringTok{'medium'}\NormalTok{)}


\CommentTok{# YOUR TURN: }
\CommentTok{# Create a factor variable education that has four categories: "High School", "College", "Masters", "Doctorate"}
\NormalTok{education <-}\StringTok{ }\KeywordTok{c}\NormalTok{(}\StringTok{"High School"}\NormalTok{, }\StringTok{"College"}\NormalTok{, }\StringTok{"Masters"}\NormalTok{, }\StringTok{"Doctorate"}\NormalTok{)}
\NormalTok{degree <-}\StringTok{ }\KeywordTok{factor}\NormalTok{(education, }\DataTypeTok{ordered =} \OtherTok{TRUE}\NormalTok{)}
\NormalTok{degree}
\end{Highlighting}
\end{Shaded}

\begin{verbatim}
## [1] High School College     Masters     Doctorate  
## Levels: College < Doctorate < High School < Masters
\end{verbatim}

\hypertarget{r-functions}{%
\subsubsection{R functions}\label{r-functions}}

\begin{Shaded}
\begin{Highlighting}[]
\CommentTok{# call the sqrt() function, passing it an argument of 25}
\KeywordTok{sqrt}\NormalTok{(}\DecValTok{25}\NormalTok{)}
\end{Highlighting}
\end{Shaded}

\begin{verbatim}
## [1] 5
\end{verbatim}

\begin{Shaded}
\begin{Highlighting}[]
\CommentTok{# call the print function, passing it "IMT 573" as an argument}
\KeywordTok{print}\NormalTok{(}\StringTok{"IMT 573"}\NormalTok{)}
\end{Highlighting}
\end{Shaded}

\begin{verbatim}
## [1] "IMT 573"
\end{verbatim}

\begin{Shaded}
\begin{Highlighting}[]
\CommentTok{#printing using cat function}
\KeywordTok{cat}\NormalTok{(}\StringTok{"value = "}\NormalTok{, a)}
\end{Highlighting}
\end{Shaded}

\begin{verbatim}
## value =  1 2 3 4 5 6 7 8 9 10 11 12 13 14 15 16 17 18 19 20 21 22 23 24 25
\end{verbatim}

\begin{Shaded}
\begin{Highlighting}[]
\CommentTok{#min and max function taking multiple arguments}
\KeywordTok{min}\NormalTok{(a)}
\end{Highlighting}
\end{Shaded}

\begin{verbatim}
## [1] 1
\end{verbatim}

\begin{Shaded}
\begin{Highlighting}[]
\KeywordTok{max}\NormalTok{(a)}
\end{Highlighting}
\end{Shaded}

\begin{verbatim}
## [1] 25
\end{verbatim}

\begin{Shaded}
\begin{Highlighting}[]
\CommentTok{#function to return upper case}
\KeywordTok{toupper}\NormalTok{(}\StringTok{"s"}\NormalTok{)}
\end{Highlighting}
\end{Shaded}

\begin{verbatim}
## [1] "S"
\end{verbatim}

\begin{Shaded}
\begin{Highlighting}[]
\CommentTok{#Write a function of your name. Let's see how it works through an example}
\CommentTok{#write a function to combine first and last name}

\NormalTok{make_fullname <-}\StringTok{ }\ControlFlowTok{function}\NormalTok{(firstname, lastname) \{}
  \CommentTok{# function body}
\NormalTok{  fullname <-}\StringTok{ }\KeywordTok{paste}\NormalTok{(firstname, lastname)}
  
  \CommentTok{#return the value}
  \KeywordTok{return}\NormalTok{(fullname)}
\NormalTok{\}}

\CommentTok{#call the function}
\NormalTok{some_name =}\StringTok{ }\KeywordTok{make_fullname}\NormalTok{(}\StringTok{'John'}\NormalTok{, }\StringTok{'Doe'}\NormalTok{)}

\CommentTok{### R functions: YOUR TURN - }\AlertTok{TODO}
\CommentTok{### Write a function to calculate area of a rectangle}
\NormalTok{rectangle_area <-}\StringTok{ }\ControlFlowTok{function}\NormalTok{(x, y) \{}
\NormalTok{area <-}\StringTok{ }\NormalTok{x }\OperatorTok{*}\StringTok{ }\NormalTok{y}
\KeywordTok{return}\NormalTok{(area)}
\NormalTok{\}}

\NormalTok{some_area =}\StringTok{ }\KeywordTok{rectangle_area}\NormalTok{(}\DecValTok{2}\NormalTok{, }\DecValTok{2}\NormalTok{)}
\NormalTok{some_area}
\end{Highlighting}
\end{Shaded}

\begin{verbatim}
## [1] 4
\end{verbatim}

\hypertarget{data-frames---act-like-tables-where-data-is-organized-into-rows-and-columns}{%
\subsubsection{Data frames - act like tables where data is organized
into rows and
columns}\label{data-frames---act-like-tables-where-data-is-organized-into-rows-and-columns}}

\begin{Shaded}
\begin{Highlighting}[]
\CommentTok{# creating a dataframe by passing vectors to the `data.frame()` function}

\CommentTok{# a vector of names}
\NormalTok{name <-}\StringTok{ }\KeywordTok{c}\NormalTok{(}\StringTok{"Alice"}\NormalTok{, }\StringTok{"Bob"}\NormalTok{, }\StringTok{"Chris"}\NormalTok{, }\StringTok{"Diya"}\NormalTok{, }\StringTok{"Emma"}\NormalTok{)}
\CommentTok{# A vector of heights}
\NormalTok{heights <-}\StringTok{ }\KeywordTok{c}\NormalTok{(}\FloatTok{5.5}\NormalTok{, }\DecValTok{6}\NormalTok{, }\FloatTok{5.3}\NormalTok{, }\FloatTok{5.8}\NormalTok{, }\FloatTok{5.9}\NormalTok{)}
\NormalTok{weights <-}\StringTok{ }\KeywordTok{c}\NormalTok{(}\DecValTok{100}\NormalTok{, }\DecValTok{170}\NormalTok{, }\DecValTok{150}\NormalTok{, }\DecValTok{120}\NormalTok{, }\DecValTok{155}\NormalTok{)}

\CommentTok{#Combine the vectors into a data frame}
\CommentTok{# Note the names of the variables become the names of the columns in the dataframe}
\NormalTok{people <-}\StringTok{ }\KeywordTok{data.frame}\NormalTok{(name, heights, weights)}

\CommentTok{# to create row.names}
\NormalTok{people2 <-}\StringTok{ }\KeywordTok{data.frame}\NormalTok{(name, heights, weights, }\DataTypeTok{row.names =} \DecValTok{1}\NormalTok{)}

\CommentTok{# YOUR TURN}
\CommentTok{# build an employee data frame of 5 employees with 3 columns: income, manager (T/F), empid }
\NormalTok{name2 <-}\StringTok{ }\KeywordTok{c}\NormalTok{(}\StringTok{"Alex"}\NormalTok{, }\StringTok{"Anna"}\NormalTok{, }\StringTok{"Jean"}\NormalTok{, }\StringTok{"Zack"}\NormalTok{, }\StringTok{"Daniel"}\NormalTok{)}
\NormalTok{income <-}\StringTok{ }\KeywordTok{c}\NormalTok{(}\DecValTok{2200}\NormalTok{, }\DecValTok{2300}\NormalTok{, }\DecValTok{5400}\NormalTok{, }\DecValTok{6800}\NormalTok{, }\DecValTok{10000}\NormalTok{)}
\NormalTok{manager <-}\StringTok{ }\KeywordTok{c}\NormalTok{(}\OtherTok{TRUE}\NormalTok{, }\OtherTok{FALSE}\NormalTok{, }\OtherTok{FALSE}\NormalTok{, }\OtherTok{TRUE}\NormalTok{, }\OtherTok{FALSE}\NormalTok{)}
\NormalTok{empid <-}\StringTok{ }\KeywordTok{c}\NormalTok{(}\DecValTok{1}\OperatorTok{:}\DecValTok{5}\NormalTok{)}

\NormalTok{employees <-}\StringTok{ }\KeywordTok{data.frame}\NormalTok{(name2, income, manager, empid)}
\NormalTok{employees}
\end{Highlighting}
\end{Shaded}

\begin{verbatim}
##    name2 income manager empid
## 1   Alex   2200    TRUE     1
## 2   Anna   2300   FALSE     2
## 3   Jean   5400   FALSE     3
## 4   Zack   6800    TRUE     4
## 5 Daniel  10000   FALSE     5
\end{verbatim}

\begin{Shaded}
\begin{Highlighting}[]
\CommentTok{# elements by row and column name}
\NormalTok{people2[}\StringTok{'Alice'}\NormalTok{, }\StringTok{'heights'}\NormalTok{]}
\end{Highlighting}
\end{Shaded}

\begin{verbatim}
## [1] 5.5
\end{verbatim}

\hypertarget{read-and-write-data}{%
\subsubsection{Read and Write data}\label{read-and-write-data}}

\textbf{Working Directory} When working with .csv files, the read.csv()
function takes as an argument a path to a file. You need to make sure
you have the correct path. To check your current working directory using
the R function \textbf{getwd()}

\begin{Shaded}
\begin{Highlighting}[]
\CommentTok{#read titanic.csv data. Download data from Canvas}
\NormalTok{data <-}\StringTok{ }\KeywordTok{read.csv}\NormalTok{(}\StringTok{'titanic.csv'}\NormalTok{)}

\CommentTok{#check the type of data}
\KeywordTok{typeof}\NormalTok{(data)}
\end{Highlighting}
\end{Shaded}

\begin{verbatim}
## [1] "list"
\end{verbatim}

\begin{Shaded}
\begin{Highlighting}[]
\CommentTok{#check additional structure and type in the data}


\CommentTok{# inspect the data - look at top and bottom}
\KeywordTok{head}\NormalTok{(data)}
\end{Highlighting}
\end{Shaded}

\begin{verbatim}
##   pclass survived                                            name    sex
## 1      1        1                   Allen, Miss. Elisabeth Walton female
## 2      1        1                  Allison, Master. Hudson Trevor   male
## 3      1        0                    Allison, Miss. Helen Loraine female
## 4      1        0            Allison, Mr. Hudson Joshua Creighton   male
## 5      1        0 Allison, Mrs. Hudson J C (Bessie Waldo Daniels) female
## 6      1        1                             Anderson, Mr. Harry   male
##       age sibsp parch ticket     fare   cabin embarked boat body
## 1 29.0000     0     0  24160 211.3375      B5        S    2   NA
## 2  0.9167     1     2 113781 151.5500 C22 C26        S   11   NA
## 3  2.0000     1     2 113781 151.5500 C22 C26        S        NA
## 4 30.0000     1     2 113781 151.5500 C22 C26        S       135
## 5 25.0000     1     2 113781 151.5500 C22 C26        S        NA
## 6 48.0000     0     0  19952  26.5500     E12        S    3   NA
##                         home.dest
## 1                    St Louis, MO
## 2 Montreal, PQ / Chesterville, ON
## 3 Montreal, PQ / Chesterville, ON
## 4 Montreal, PQ / Chesterville, ON
## 5 Montreal, PQ / Chesterville, ON
## 6                    New York, NY
\end{verbatim}

\begin{Shaded}
\begin{Highlighting}[]
\KeywordTok{tail}\NormalTok{(data)}
\end{Highlighting}
\end{Shaded}

\begin{verbatim}
##      pclass survived                      name    sex  age sibsp parch ticket
## 1304      3        0     Yousseff, Mr. Gerious   male   NA     0     0   2627
## 1305      3        0      Zabour, Miss. Hileni female 14.5     1     0   2665
## 1306      3        0     Zabour, Miss. Thamine female   NA     1     0   2665
## 1307      3        0 Zakarian, Mr. Mapriededer   male 26.5     0     0   2656
## 1308      3        0       Zakarian, Mr. Ortin   male 27.0     0     0   2670
## 1309      3        0        Zimmerman, Mr. Leo   male 29.0     0     0 315082
##         fare cabin embarked boat body home.dest
## 1304 14.4583              C        NA          
## 1305 14.4542              C       328          
## 1306 14.4542              C        NA          
## 1307  7.2250              C       304          
## 1308  7.2250              C        NA          
## 1309  7.8750              S        NA
\end{verbatim}

\hypertarget{finding-built-in-data-sets}{%
\subsubsection{Finding built-in data
sets}\label{finding-built-in-data-sets}}

\hypertarget{elementary-visualization}{%
\subsubsection{Elementary
visualization}\label{elementary-visualization}}

\end{document}
