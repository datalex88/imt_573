% Options for packages loaded elsewhere
\PassOptionsToPackage{unicode}{hyperref}
\PassOptionsToPackage{hyphens}{url}
%
\documentclass[
]{article}
\usepackage{lmodern}
\usepackage{amssymb,amsmath}
\usepackage{ifxetex,ifluatex}
\ifnum 0\ifxetex 1\fi\ifluatex 1\fi=0 % if pdftex
  \usepackage[T1]{fontenc}
  \usepackage[utf8]{inputenc}
  \usepackage{textcomp} % provide euro and other symbols
\else % if luatex or xetex
  \usepackage{unicode-math}
  \defaultfontfeatures{Scale=MatchLowercase}
  \defaultfontfeatures[\rmfamily]{Ligatures=TeX,Scale=1}
\fi
% Use upquote if available, for straight quotes in verbatim environments
\IfFileExists{upquote.sty}{\usepackage{upquote}}{}
\IfFileExists{microtype.sty}{% use microtype if available
  \usepackage[]{microtype}
  \UseMicrotypeSet[protrusion]{basicmath} % disable protrusion for tt fonts
}{}
\makeatletter
\@ifundefined{KOMAClassName}{% if non-KOMA class
  \IfFileExists{parskip.sty}{%
    \usepackage{parskip}
  }{% else
    \setlength{\parindent}{0pt}
    \setlength{\parskip}{6pt plus 2pt minus 1pt}}
}{% if KOMA class
  \KOMAoptions{parskip=half}}
\makeatother
\usepackage{xcolor}
\IfFileExists{xurl.sty}{\usepackage{xurl}}{} % add URL line breaks if available
\IfFileExists{bookmark.sty}{\usepackage{bookmark}}{\usepackage{hyperref}}
\hypersetup{
  pdftitle={IMT 573: Lab 2 - Exploring Data},
  pdfauthor={Alexander Davis},
  hidelinks,
  pdfcreator={LaTeX via pandoc}}
\urlstyle{same} % disable monospaced font for URLs
\usepackage[margin=1in]{geometry}
\usepackage{color}
\usepackage{fancyvrb}
\newcommand{\VerbBar}{|}
\newcommand{\VERB}{\Verb[commandchars=\\\{\}]}
\DefineVerbatimEnvironment{Highlighting}{Verbatim}{commandchars=\\\{\}}
% Add ',fontsize=\small' for more characters per line
\usepackage{framed}
\definecolor{shadecolor}{RGB}{248,248,248}
\newenvironment{Shaded}{\begin{snugshade}}{\end{snugshade}}
\newcommand{\AlertTok}[1]{\textcolor[rgb]{0.94,0.16,0.16}{#1}}
\newcommand{\AnnotationTok}[1]{\textcolor[rgb]{0.56,0.35,0.01}{\textbf{\textit{#1}}}}
\newcommand{\AttributeTok}[1]{\textcolor[rgb]{0.77,0.63,0.00}{#1}}
\newcommand{\BaseNTok}[1]{\textcolor[rgb]{0.00,0.00,0.81}{#1}}
\newcommand{\BuiltInTok}[1]{#1}
\newcommand{\CharTok}[1]{\textcolor[rgb]{0.31,0.60,0.02}{#1}}
\newcommand{\CommentTok}[1]{\textcolor[rgb]{0.56,0.35,0.01}{\textit{#1}}}
\newcommand{\CommentVarTok}[1]{\textcolor[rgb]{0.56,0.35,0.01}{\textbf{\textit{#1}}}}
\newcommand{\ConstantTok}[1]{\textcolor[rgb]{0.00,0.00,0.00}{#1}}
\newcommand{\ControlFlowTok}[1]{\textcolor[rgb]{0.13,0.29,0.53}{\textbf{#1}}}
\newcommand{\DataTypeTok}[1]{\textcolor[rgb]{0.13,0.29,0.53}{#1}}
\newcommand{\DecValTok}[1]{\textcolor[rgb]{0.00,0.00,0.81}{#1}}
\newcommand{\DocumentationTok}[1]{\textcolor[rgb]{0.56,0.35,0.01}{\textbf{\textit{#1}}}}
\newcommand{\ErrorTok}[1]{\textcolor[rgb]{0.64,0.00,0.00}{\textbf{#1}}}
\newcommand{\ExtensionTok}[1]{#1}
\newcommand{\FloatTok}[1]{\textcolor[rgb]{0.00,0.00,0.81}{#1}}
\newcommand{\FunctionTok}[1]{\textcolor[rgb]{0.00,0.00,0.00}{#1}}
\newcommand{\ImportTok}[1]{#1}
\newcommand{\InformationTok}[1]{\textcolor[rgb]{0.56,0.35,0.01}{\textbf{\textit{#1}}}}
\newcommand{\KeywordTok}[1]{\textcolor[rgb]{0.13,0.29,0.53}{\textbf{#1}}}
\newcommand{\NormalTok}[1]{#1}
\newcommand{\OperatorTok}[1]{\textcolor[rgb]{0.81,0.36,0.00}{\textbf{#1}}}
\newcommand{\OtherTok}[1]{\textcolor[rgb]{0.56,0.35,0.01}{#1}}
\newcommand{\PreprocessorTok}[1]{\textcolor[rgb]{0.56,0.35,0.01}{\textit{#1}}}
\newcommand{\RegionMarkerTok}[1]{#1}
\newcommand{\SpecialCharTok}[1]{\textcolor[rgb]{0.00,0.00,0.00}{#1}}
\newcommand{\SpecialStringTok}[1]{\textcolor[rgb]{0.31,0.60,0.02}{#1}}
\newcommand{\StringTok}[1]{\textcolor[rgb]{0.31,0.60,0.02}{#1}}
\newcommand{\VariableTok}[1]{\textcolor[rgb]{0.00,0.00,0.00}{#1}}
\newcommand{\VerbatimStringTok}[1]{\textcolor[rgb]{0.31,0.60,0.02}{#1}}
\newcommand{\WarningTok}[1]{\textcolor[rgb]{0.56,0.35,0.01}{\textbf{\textit{#1}}}}
\usepackage{graphicx,grffile}
\makeatletter
\def\maxwidth{\ifdim\Gin@nat@width>\linewidth\linewidth\else\Gin@nat@width\fi}
\def\maxheight{\ifdim\Gin@nat@height>\textheight\textheight\else\Gin@nat@height\fi}
\makeatother
% Scale images if necessary, so that they will not overflow the page
% margins by default, and it is still possible to overwrite the defaults
% using explicit options in \includegraphics[width, height, ...]{}
\setkeys{Gin}{width=\maxwidth,height=\maxheight,keepaspectratio}
% Set default figure placement to htbp
\makeatletter
\def\fps@figure{htbp}
\makeatother
\setlength{\emergencystretch}{3em} % prevent overfull lines
\providecommand{\tightlist}{%
  \setlength{\itemsep}{0pt}\setlength{\parskip}{0pt}}
\setcounter{secnumdepth}{-\maxdimen} % remove section numbering

\title{IMT 573: Lab 2 - Exploring Data}
\author{Alexander Davis}
\date{Thursday, October 08, 2020}

\begin{document}
\maketitle

Collaborators:

\begin{enumerate}
\def\labelenumi{\arabic{enumi}.}
\item
\item
  \ldots{}
\end{enumerate}

\hypertarget{objectives}{%
\subsubsection{Objectives}\label{objectives}}

In this lab we will dive right in and explore a found dataset. Our aim
is to practice getting to know our data. We will follow the steps of
exploratory data analysis in this endeavor. This demo will give you an
introduction to the very popular data visualization package
\texttt{ggplot}. We will start with the basics today, and see more of
this particular tool later on in the course.

\begin{Shaded}
\begin{Highlighting}[]
\CommentTok{# Load some helpful libraries for this course}
\KeywordTok{library}\NormalTok{(tidyverse)}
\KeywordTok{library}\NormalTok{(ggplot2)}
\end{Highlighting}
\end{Shaded}

\hypertarget{data-background}{%
\subsubsection{Data Background}\label{data-background}}

The sinking of the RMS Titanic\footnote{\url{https://en.wikipedia.org/wiki/RMS_Titanic}}
is a notable historical event. The RMS Titanic was a British passenger
liner that sank in the North Atlantic Ocean in the early morning of 15
April 1912, after colliding with an iceberg during her maiden voyage
from Southampton to New York City. Of the 2,224 passengers and crew
aboard, more than 1,500 died in the sinking, making it one of the
deadliest commercial peacetime maritime disasters in modern history.

The disaster was greeted with worldwide shock and outrage at the huge
loss of life and the regulatory and operational failures that had led to
it. Public inquiries in Britain and the United States led to major
improvements in maritime safety. One of their most important legacies
was the establishment in 1914 of the International Convention for the
Safety of Life at Sea (SOLAS)\footnote{\url{https://en.wikipedia.org/wiki/International_Convention_for_the_Safety_of_Life_at_Sea}},
which still governs maritime safety today. Additionally, several new
wireless regulations were passed around the world in an effort to learn
from the many missteps in wireless communications---which could have
saved many more passengers.

The data we will explore in this lab were originally collected by the
British Board of Trade in their investigation of the sinking. You can
download these data in CSV format from Canvas. Researchers should note
that there is not complete agreement among primary sources as to the
exact numbers on board, rescued, or lost.

\hypertarget{formulating-a-question}{%
\subsubsection{Formulating a Question}\label{formulating-a-question}}

Today, we will consider two questions in our exploration:

\begin{itemize}
\tightlist
\item
  Who were the Titanic passengers? What characteristics did they have?
\item
  What passenger characteristics or other factors are associated with
  survival?
\end{itemize}

\hypertarget{read-and-inspect-data}{%
\subsubsection{Read and Inspect Data}\label{read-and-inspect-data}}

To begin, we need to load the Titanic dataset into R. You can do so by
executing the following code.

\begin{Shaded}
\begin{Highlighting}[]
\KeywordTok{library}\NormalTok{(titanic)}
\NormalTok{titanic_data <-}\StringTok{ }\KeywordTok{data.frame}\NormalTok{(titanic_train)}
\end{Highlighting}
\end{Shaded}

Next, we want to inspect our data. We don't want to assume that our data
is in exactly as we expect it to be after reading it into R. It is
helpful to inspect the data object, confirming to looks as expected.

Try editing to following code chunk to look at the top and bottom of
your data frame. Perform any other inspection operations you deem
necessary. Do you observe anything concerning?

\begin{Shaded}
\begin{Highlighting}[]
\KeywordTok{head}\NormalTok{(titanic_data)}
\end{Highlighting}
\end{Shaded}

\begin{verbatim}
##   PassengerId Survived Pclass
## 1           1        0      3
## 2           2        1      1
## 3           3        1      3
## 4           4        1      1
## 5           5        0      3
## 6           6        0      3
##                                                  Name    Sex Age SibSp Parch
## 1                             Braund, Mr. Owen Harris   male  22     1     0
## 2 Cumings, Mrs. John Bradley (Florence Briggs Thayer) female  38     1     0
## 3                              Heikkinen, Miss. Laina female  26     0     0
## 4        Futrelle, Mrs. Jacques Heath (Lily May Peel) female  35     1     0
## 5                            Allen, Mr. William Henry   male  35     0     0
## 6                                    Moran, Mr. James   male  NA     0     0
##             Ticket    Fare Cabin Embarked
## 1        A/5 21171  7.2500              S
## 2         PC 17599 71.2833   C85        C
## 3 STON/O2. 3101282  7.9250              S
## 4           113803 53.1000  C123        S
## 5           373450  8.0500              S
## 6           330877  8.4583              Q
\end{verbatim}

\begin{Shaded}
\begin{Highlighting}[]
\KeywordTok{tail}\NormalTok{(titanic_data)}
\end{Highlighting}
\end{Shaded}

\begin{verbatim}
##     PassengerId Survived Pclass                                     Name    Sex
## 886         886        0      3     Rice, Mrs. William (Margaret Norton) female
## 887         887        0      2                    Montvila, Rev. Juozas   male
## 888         888        1      1             Graham, Miss. Margaret Edith female
## 889         889        0      3 Johnston, Miss. Catherine Helen "Carrie" female
## 890         890        1      1                    Behr, Mr. Karl Howell   male
## 891         891        0      3                      Dooley, Mr. Patrick   male
##     Age SibSp Parch     Ticket   Fare Cabin Embarked
## 886  39     0     5     382652 29.125              Q
## 887  27     0     0     211536 13.000              S
## 888  19     0     0     112053 30.000   B42        S
## 889  NA     1     2 W./C. 6607 23.450              S
## 890  26     0     0     111369 30.000  C148        C
## 891  32     0     0     370376  7.750              Q
\end{verbatim}

\textcolor{red}{Solution: }\emph{}

Think about the variables in this data as they are defined. Which
variables might you want to re-cast to be the appropriate data type in
R? \emph{Hint: Categorical variables are better suited as factors. You
might want to check \texttt{as.factor}}

\textcolor{red}{Solution: }\emph{}

Transform the data type of variables you identify as improperly cast.

\begin{Shaded}
\begin{Highlighting}[]
\NormalTok{titanic_data}\OperatorTok{$}\NormalTok{survived <-}\StringTok{ }\KeywordTok{as.factor}\NormalTok{(titanic_data}\OperatorTok{$}\NormalTok{Survived)}
\end{Highlighting}
\end{Shaded}

\hypertarget{trying-the-easy-solution-first}{%
\subsubsection{Trying the Easy Solution
First}\label{trying-the-easy-solution-first}}

First, we want to explore who the passengers aboard the Titanic were.
There are many ways we might go about this. Consider for example trying
to understand the ages of passengers. We can create a basic
visualization to help us understand the distributions of age for Titanic
passengers.

You could also explore other characteristics, like passenger class,
gender.

\begin{Shaded}
\begin{Highlighting}[]
\NormalTok{passenger_age <-}\StringTok{ }\KeywordTok{data.frame}\NormalTok{(titanic_data[}\DecValTok{1}\NormalTok{],titanic_data[}\DecValTok{6}\NormalTok{])}
\CommentTok{##passenger_age}
\KeywordTok{ggplot}\NormalTok{(passenger_age, }\KeywordTok{aes}\NormalTok{(}\DataTypeTok{x=}\NormalTok{Age)) }\OperatorTok{+}\StringTok{ }\KeywordTok{geom_histogram}\NormalTok{(}\DataTypeTok{color=}\StringTok{"black"}\NormalTok{, }\DataTypeTok{fill=}\StringTok{"white"}\NormalTok{)}
\end{Highlighting}
\end{Shaded}

\begin{verbatim}
## `stat_bin()` using `bins = 30`. Pick better value with `binwidth`.
\end{verbatim}

\begin{verbatim}
## Warning: Removed 177 rows containing non-finite values (stat_bin).
\end{verbatim}

\includegraphics{lab2_exploring_data_files/figure-latex/exploreData-1.pdf}

We might go further to look at how passenger age might be related to
survival (or whichever other characteristic you are interested in).

\begin{Shaded}
\begin{Highlighting}[]
\NormalTok{passenger_age_survival <-}\StringTok{ }\KeywordTok{data.frame}\NormalTok{(passenger_age, titanic_data[}\DecValTok{2}\NormalTok{])}
\CommentTok{#passenger_age_survival}
\KeywordTok{ggplot}\NormalTok{(passenger_age_survival, }\KeywordTok{aes}\NormalTok{(}\DataTypeTok{x=}\NormalTok{Age)) }\OperatorTok{+}\StringTok{ }\KeywordTok{geom_histogram}\NormalTok{(}\DataTypeTok{color=}\StringTok{"black"}\NormalTok{, }\DataTypeTok{fill=}\StringTok{"white"}\NormalTok{) }\OperatorTok{+}\StringTok{ }\KeywordTok{facet_grid}\NormalTok{(}\KeywordTok{vars}\NormalTok{(Survived))}
\end{Highlighting}
\end{Shaded}

\begin{verbatim}
## `stat_bin()` using `bins = 30`. Pick better value with `binwidth`.
\end{verbatim}

\begin{verbatim}
## Warning: Removed 177 rows containing non-finite values (stat_bin).
\end{verbatim}

\includegraphics{lab2_exploring_data_files/figure-latex/explore data with survival-1.pdf}

Do you like the above figure? Why or why not? Produce a new figure that
you think does a better job of helping you explore the association
between passenger age and survival.

\emph{I think this does an okay job at exploring some association
between age and survival. You can pick out that a significant portion of
younger individuals survived in the below histogram versus the above
histogram.}

\textcolor{red}{Solution: }\emph{}

Identify one additional data feature you want to explore. Produce one
visualization that explore this feature. Describe why you think this is
interesting and what you find.

\emph{I would take a look at class and how that could determine a
possible survival rate.} \textcolor{red}{Solution: }\emph{}

\hypertarget{what-next}{%
\subsubsection{What Next?}\label{what-next}}

Consider the exploratory analysis you just completed. What would you do
next? Can you challenge your first approach and come up with additional
questions and solutions? See hint \footnote{\url{https://bio304-class.github.io/bio304-fall2017/data-story-titanic.html}}.

\textcolor{red}{Solution: }\emph{}

\end{document}
