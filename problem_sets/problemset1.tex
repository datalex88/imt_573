% Options for packages loaded elsewhere
\PassOptionsToPackage{unicode}{hyperref}
\PassOptionsToPackage{hyphens}{url}
%
\documentclass[
]{article}
\usepackage{lmodern}
\usepackage{amssymb,amsmath}
\usepackage{ifxetex,ifluatex}
\ifnum 0\ifxetex 1\fi\ifluatex 1\fi=0 % if pdftex
  \usepackage[T1]{fontenc}
  \usepackage[utf8]{inputenc}
  \usepackage{textcomp} % provide euro and other symbols
\else % if luatex or xetex
  \usepackage{unicode-math}
  \defaultfontfeatures{Scale=MatchLowercase}
  \defaultfontfeatures[\rmfamily]{Ligatures=TeX,Scale=1}
\fi
% Use upquote if available, for straight quotes in verbatim environments
\IfFileExists{upquote.sty}{\usepackage{upquote}}{}
\IfFileExists{microtype.sty}{% use microtype if available
  \usepackage[]{microtype}
  \UseMicrotypeSet[protrusion]{basicmath} % disable protrusion for tt fonts
}{}
\makeatletter
\@ifundefined{KOMAClassName}{% if non-KOMA class
  \IfFileExists{parskip.sty}{%
    \usepackage{parskip}
  }{% else
    \setlength{\parindent}{0pt}
    \setlength{\parskip}{6pt plus 2pt minus 1pt}}
}{% if KOMA class
  \KOMAoptions{parskip=half}}
\makeatother
\usepackage{xcolor}
\IfFileExists{xurl.sty}{\usepackage{xurl}}{} % add URL line breaks if available
\IfFileExists{bookmark.sty}{\usepackage{bookmark}}{\usepackage{hyperref}}
\hypersetup{
  pdftitle={IMT 573: Problem Set 1 - Exploring Data},
  pdfauthor={Alexander Davis},
  hidelinks,
  pdfcreator={LaTeX via pandoc}}
\urlstyle{same} % disable monospaced font for URLs
\usepackage[margin=1in]{geometry}
\usepackage{color}
\usepackage{fancyvrb}
\newcommand{\VerbBar}{|}
\newcommand{\VERB}{\Verb[commandchars=\\\{\}]}
\DefineVerbatimEnvironment{Highlighting}{Verbatim}{commandchars=\\\{\}}
% Add ',fontsize=\small' for more characters per line
\usepackage{framed}
\definecolor{shadecolor}{RGB}{248,248,248}
\newenvironment{Shaded}{\begin{snugshade}}{\end{snugshade}}
\newcommand{\AlertTok}[1]{\textcolor[rgb]{0.94,0.16,0.16}{#1}}
\newcommand{\AnnotationTok}[1]{\textcolor[rgb]{0.56,0.35,0.01}{\textbf{\textit{#1}}}}
\newcommand{\AttributeTok}[1]{\textcolor[rgb]{0.77,0.63,0.00}{#1}}
\newcommand{\BaseNTok}[1]{\textcolor[rgb]{0.00,0.00,0.81}{#1}}
\newcommand{\BuiltInTok}[1]{#1}
\newcommand{\CharTok}[1]{\textcolor[rgb]{0.31,0.60,0.02}{#1}}
\newcommand{\CommentTok}[1]{\textcolor[rgb]{0.56,0.35,0.01}{\textit{#1}}}
\newcommand{\CommentVarTok}[1]{\textcolor[rgb]{0.56,0.35,0.01}{\textbf{\textit{#1}}}}
\newcommand{\ConstantTok}[1]{\textcolor[rgb]{0.00,0.00,0.00}{#1}}
\newcommand{\ControlFlowTok}[1]{\textcolor[rgb]{0.13,0.29,0.53}{\textbf{#1}}}
\newcommand{\DataTypeTok}[1]{\textcolor[rgb]{0.13,0.29,0.53}{#1}}
\newcommand{\DecValTok}[1]{\textcolor[rgb]{0.00,0.00,0.81}{#1}}
\newcommand{\DocumentationTok}[1]{\textcolor[rgb]{0.56,0.35,0.01}{\textbf{\textit{#1}}}}
\newcommand{\ErrorTok}[1]{\textcolor[rgb]{0.64,0.00,0.00}{\textbf{#1}}}
\newcommand{\ExtensionTok}[1]{#1}
\newcommand{\FloatTok}[1]{\textcolor[rgb]{0.00,0.00,0.81}{#1}}
\newcommand{\FunctionTok}[1]{\textcolor[rgb]{0.00,0.00,0.00}{#1}}
\newcommand{\ImportTok}[1]{#1}
\newcommand{\InformationTok}[1]{\textcolor[rgb]{0.56,0.35,0.01}{\textbf{\textit{#1}}}}
\newcommand{\KeywordTok}[1]{\textcolor[rgb]{0.13,0.29,0.53}{\textbf{#1}}}
\newcommand{\NormalTok}[1]{#1}
\newcommand{\OperatorTok}[1]{\textcolor[rgb]{0.81,0.36,0.00}{\textbf{#1}}}
\newcommand{\OtherTok}[1]{\textcolor[rgb]{0.56,0.35,0.01}{#1}}
\newcommand{\PreprocessorTok}[1]{\textcolor[rgb]{0.56,0.35,0.01}{\textit{#1}}}
\newcommand{\RegionMarkerTok}[1]{#1}
\newcommand{\SpecialCharTok}[1]{\textcolor[rgb]{0.00,0.00,0.00}{#1}}
\newcommand{\SpecialStringTok}[1]{\textcolor[rgb]{0.31,0.60,0.02}{#1}}
\newcommand{\StringTok}[1]{\textcolor[rgb]{0.31,0.60,0.02}{#1}}
\newcommand{\VariableTok}[1]{\textcolor[rgb]{0.00,0.00,0.00}{#1}}
\newcommand{\VerbatimStringTok}[1]{\textcolor[rgb]{0.31,0.60,0.02}{#1}}
\newcommand{\WarningTok}[1]{\textcolor[rgb]{0.56,0.35,0.01}{\textbf{\textit{#1}}}}
\usepackage{graphicx,grffile}
\makeatletter
\def\maxwidth{\ifdim\Gin@nat@width>\linewidth\linewidth\else\Gin@nat@width\fi}
\def\maxheight{\ifdim\Gin@nat@height>\textheight\textheight\else\Gin@nat@height\fi}
\makeatother
% Scale images if necessary, so that they will not overflow the page
% margins by default, and it is still possible to overwrite the defaults
% using explicit options in \includegraphics[width, height, ...]{}
\setkeys{Gin}{width=\maxwidth,height=\maxheight,keepaspectratio}
% Set default figure placement to htbp
\makeatletter
\def\fps@figure{htbp}
\makeatother
\setlength{\emergencystretch}{3em} % prevent overfull lines
\providecommand{\tightlist}{%
  \setlength{\itemsep}{0pt}\setlength{\parskip}{0pt}}
\setcounter{secnumdepth}{-\maxdimen} % remove section numbering

\title{IMT 573: Problem Set 1 - Exploring Data}
\author{Alexander Davis}
\date{Due: Tuesday, October 13, 2020 by 9am PT}

\begin{document}
\maketitle

\hypertarget{collaborators}{%
\subparagraph{Collaborators: }\label{collaborators}}

\hypertarget{instructions}{%
\subparagraph{Instructions:}\label{instructions}}

Before beginning this assignment, please ensure you have access to R and
RStudio.

\begin{enumerate}
\def\labelenumi{\arabic{enumi}.}
\item
  Download the \texttt{problemset1.Rmd} file from Canvas. Open
  \texttt{problemset1.Rmd} in RStudio and supply your solutions to the
  assignment by editing \texttt{problemset1.Rmd}.
\item
  Replace the ``Insert Your Name Here'' text in the \texttt{author:}
  field with your own full name. Any collaborators must be listed on the
  top of your assignment. Collaboration shouldn't be confused with group
  project work (where each person does a part of the project). Working
  on problem sets should be your individual contribution. More on that
  in point 8.
\item
  Be sure to include well-documented (e.g.~commented) code chucks,
  figures, and clearly written text chunk explanations as necessary. Any
  figures should be clearly labeled and appropriately referenced within
  the text. Be sure that each visualization adds value to your written
  explanation; avoid redundancy -- you do not need four different
  visualizations of the same pattern.
\item
  All materials and resources that you use (with the exception of
  lecture slides) must be appropriately referenced within your
  assignment. In particular, note that Stack Overflow is licenses as
  Creative Commons (CC-BY-SA). This means you have to attribute any code
  you refer from SO.
\item
  Partial credit will be awarded for each question for which a serious
  attempt at finding an answer has been shown. But please \textbf{DO
  NOT} submit pages and pages of hard-to-read code and attempts that is
  impossible to grade. That is, avoid redundancy. Remember that one of
  the key goals of a data scientist is to produce coherent reports that
  others can easily follow. Students are \emph{strongly} encouraged to
  attempt each question and to document their reasoning process even if
  they cannot find the correct answer. If you would like to include R
  code to show this process, but it does not run without errors you can
  do so with the \texttt{eval=FALSE} option as follows:
\end{enumerate}

\begin{Shaded}
\begin{Highlighting}[]
\NormalTok{a }\OperatorTok{+}\StringTok{ }\NormalTok{b }\CommentTok{# these object dont' exist }
\CommentTok{# if you run this on its own it with give an error}
\end{Highlighting}
\end{Shaded}

\begin{enumerate}
\def\labelenumi{\arabic{enumi}.}
\setcounter{enumi}{6}
\item
  When you have completed the assignment and have \textbf{checked} that
  your code both runs in the Console and knits correctly when you click
  \texttt{Knit\ PDF}, rename the knitted PDF file to
  \texttt{ps1\_ourLastName\_YourFirstName.pdf}, and submit the PDF file
  on Canvas.
\item
  Collaboration is often fun and useful, but each student must turn in
  an individual write-up in their own words as well as code/work that is
  their own. Regardless of whether you work with others, what you turn
  in must be your own work; this includes code and interpretation of
  results. The names of all collaborators must be listed on each
  assignment. Do not copy-and-paste from other students' responses or
  code.
\end{enumerate}

\hypertarget{problem-1-basic-r-programming}{%
\paragraph{Problem 1: Basic R
Programming}\label{problem-1-basic-r-programming}}

Write a function, \texttt{calculate\_bmi} to calculate a person's body
mass index, when given two input parameters, 1). weight in pounds and 2)
height in inches.

\emph{NOTE: You would have to go to external sources to find the formula
of bmi.} In your response, before presenting your code for the function,
tell us your official reference for the BMI formulae.

\hypertarget{insert-response-first}{%
\subparagraph{Insert Response first}\label{insert-response-first}}

\hypertarget{insert-code.-your-code-should-appear-within-r-code-chunks.}{%
\subparagraph{Insert code. Your code should appear within R Code
Chunks.}\label{insert-code.-your-code-should-appear-within-r-code-chunks.}}

\begin{Shaded}
\begin{Highlighting}[]
\NormalTok{calculate_bmi <-}\StringTok{ }\ControlFlowTok{function}\NormalTok{(weight, height) \{}
\NormalTok{  bmi <-}\StringTok{ }\KeywordTok{sum}\NormalTok{(weight, height)}
  \KeywordTok{return}\NormalTok{(bmi)}
\NormalTok{\}}

\KeywordTok{calculate_bmi}\NormalTok{(}\DecValTok{5}\NormalTok{,}\DecValTok{10}\NormalTok{)}
\end{Highlighting}
\end{Shaded}

\begin{verbatim}
## [1] 15
\end{verbatim}

\hypertarget{problem-2-exploring-the-nyc-flights-data}{%
\paragraph{Problem 2: Exploring the NYC Flights
Data}\label{problem-2-exploring-the-nyc-flights-data}}

In this problem set, we will use the data on all flights that departed
NYC (i.e.~JFK, LGA or EWR) in 2013. You can find this data in the
\texttt{nycflights13} R package.

\hypertarget{setup-problem-2}{%
\subparagraph{Setup: Problem 2}\label{setup-problem-2}}

You will need, at minimum, the following R packages. The data itself
resides in package \emph{nycflights13}. You may need to install both.

\begin{Shaded}
\begin{Highlighting}[]
\CommentTok{# Load standard libraries}
\KeywordTok{library}\NormalTok{(dplyr)}
\KeywordTok{library}\NormalTok{(tidyverse)}
\KeywordTok{library}\NormalTok{(}\StringTok{'nycflights13'}\NormalTok{)}
\end{Highlighting}
\end{Shaded}

\begin{Shaded}
\begin{Highlighting}[]
\CommentTok{# Load the nycflights13 library which includes data on all}
\CommentTok{# lights departing NYC}
\KeywordTok{data}\NormalTok{(flights)}

\CommentTok{# Note the data itself is called flights, we will make it into a local df}
\CommentTok{# for readability}
\NormalTok{flights <-}\StringTok{ }\KeywordTok{tbl_df}\NormalTok{(flights)}
\end{Highlighting}
\end{Shaded}

\begin{verbatim}
## Warning: `tbl_df()` is deprecated as of dplyr 1.0.0.
## Please use `tibble::as_tibble()` instead.
## This warning is displayed once every 8 hours.
## Call `lifecycle::last_warnings()` to see where this warning was generated.
\end{verbatim}

\begin{Shaded}
\begin{Highlighting}[]
\CommentTok{# Look at the help file for information about the data}
\CommentTok{# ?flights}
\NormalTok{flights}
\end{Highlighting}
\end{Shaded}

\begin{verbatim}
## # A tibble: 336,776 x 19
##     year month   day dep_time sched_dep_time dep_delay arr_time sched_arr_time
##    <int> <int> <int>    <int>          <int>     <dbl>    <int>          <int>
##  1  2013     1     1      517            515         2      830            819
##  2  2013     1     1      533            529         4      850            830
##  3  2013     1     1      542            540         2      923            850
##  4  2013     1     1      544            545        -1     1004           1022
##  5  2013     1     1      554            600        -6      812            837
##  6  2013     1     1      554            558        -4      740            728
##  7  2013     1     1      555            600        -5      913            854
##  8  2013     1     1      557            600        -3      709            723
##  9  2013     1     1      557            600        -3      838            846
## 10  2013     1     1      558            600        -2      753            745
## # ... with 336,766 more rows, and 11 more variables: arr_delay <dbl>,
## #   carrier <chr>, flight <int>, tailnum <chr>, origin <chr>, dest <chr>,
## #   air_time <dbl>, distance <dbl>, hour <dbl>, minute <dbl>, time_hour <dttm>
\end{verbatim}

\begin{Shaded}
\begin{Highlighting}[]
\CommentTok{# summary(flights)}
\end{Highlighting}
\end{Shaded}

\hypertarget{a-importing-data}{%
\subparagraph{(a) Importing Data}\label{a-importing-data}}

Load the data and describe in a short paragraph how the data was
collected and what each variable represents.

\emph{This is the data collected over the period of a year from flights
going into and out of NYC. The variables include; YEAR, MONTH, DAY as
well as departure time (dep\_time) which would be in a time format of
5:17 for example, a scheduled departure time (shed\_dep\_time) in
similar format, departure delay (dep\_delay) in minuits either positive
or negative values depending if the flight departed early or late, an
arrival time (arr\_time) in a similar time format, a scheduled arrival
time (shed\_arr\_time), and an arrival delay (arr\_delay) which is also
in a minutes format. There is then the carrier which is just 2 letters,
the flight number (numbers), a tail number which is a combination of
letters and numbers referencing the specific plane, the origin airport
and the destination airport. Total Air\_time which is in minutes,
distance in miles, and then hour and minute column followed by
time\_hour.}

\hypertarget{b-inspecting-data}{%
\subparagraph{(b) Inspecting Data}\label{b-inspecting-data}}

Perform a basic inspection of the data and discuss what you find.
Inspections may involve asking the following questions (the list is not
inclusive, you may well ask other questions):

\begin{itemize}
\tightlist
\item
  How many distinct flights do we have in the dataset?
\item
  How many missing values are there in each variable?
\item
  Do you see any unreasonable values? \emph{Hint: Check out min, max and
  range functions.}
\end{itemize}

\begin{Shaded}
\begin{Highlighting}[]
\CommentTok{#unique(flights$carrier) ## doublecheck what 00, 9E, and B6 means.}

\CommentTok{#max(flights$distance)}

\CommentTok{#min(flights$distance) ## some possible issues with a 17 mile trip??}

\NormalTok{flight_time_wo <-}\StringTok{ }\NormalTok{flights }\OperatorTok
\StringTok{  }\KeywordTok{filter}\NormalTok{(}\OperatorTok{!}\KeywordTok{is.na}\NormalTok{(air_time), }\OperatorTok{!}\KeywordTok{is.na}\NormalTok{(air_time)) }\CommentTok{## Removing NA values... these could potentiall be canceled flights.}

\KeywordTok{min}\NormalTok{(flight_time_wo}\OperatorTok{$}\NormalTok{air_time) }\CommentTok{## Seeing a min of 20 min which could be weird.}
\end{Highlighting}
\end{Shaded}

\begin{verbatim}
## [1] 20
\end{verbatim}

\hypertarget{c-formulating-questions}{%
\subparagraph{(c) Formulating Questions}\label{c-formulating-questions}}

Consider the NYC flights data. Formulate two motivating questions you
want to explore using this data. Describe why these questions are
interesting and how you might go about answering them.

Example questions:

\begin{itemize}
\tightlist
\item
  Which airport, JFK or LGA, experience more delays?
\item
  What was the worst day to fly out?
\item
  Are there seasonal patterns
\end{itemize}

\emph{Do origin airports experience seasonal delays more than others?
This would be interesting as one could make an informed decision to
choose to fly from one airport over another to avoid delays. This kind
of data would be useful also for busiensses and/or goverment bodies to
better organize how flights are scheduled to try and optimize airport
usage or to find which airport might need some remodeling/optimization
to support increased traffic}

\emph{Which airline experiences more delays and by how much?}

\hypertarget{d-exploring-data}{%
\subparagraph{(d) Exploring Data}\label{d-exploring-data}}

For each of the questions you proposed in Problem 1c, perform an
exploratory data analysis designed to address the question. Produce
visualizations (graphics or tables) to answer your question. * You need
to explore the data from the point of view of the questions * Depending
on the question, you would need to provide precise definition. For
example, what does ``more delays'' mean. * At a minimum, you should
produce two visualizations (graphics or tables) related to each
question. Be sure to describe what the visuals show and how they speak
to your question of interest.

\begin{Shaded}
\begin{Highlighting}[]
\CommentTok{## Do origin airports experience seasonal delays more than others?}

\NormalTok{airport_data <-}\StringTok{ }\NormalTok{flights }\OperatorTok
\StringTok{  }\KeywordTok{group_by}\NormalTok{(origin, month) }\OperatorTok
\StringTok{  }\KeywordTok{summarize}\NormalTok{(}\DataTypeTok{mean =} \KeywordTok{mean}\NormalTok{(dep_delay, }\DataTypeTok{na.rm =} \OtherTok{TRUE}\NormalTok{))}
\end{Highlighting}
\end{Shaded}

\begin{verbatim}
## `summarise()` regrouping output by 'origin' (override with `.groups` argument)
\end{verbatim}

\begin{Shaded}
\begin{Highlighting}[]
\NormalTok{airport_data}
\end{Highlighting}
\end{Shaded}

\begin{verbatim}
## # A tibble: 36 x 3
## # Groups:   origin [3]
##    origin month  mean
##    <chr>  <int> <dbl>
##  1 EWR        1 14.9 
##  2 EWR        2 13.1 
##  3 EWR        3 18.1 
##  4 EWR        4 17.4 
##  5 EWR        5 15.4 
##  6 EWR        6 22.5 
##  7 EWR        7 22.0 
##  8 EWR        8 13.5 
##  9 EWR        9  7.29
## 10 EWR       10  8.64
## # ... with 26 more rows
\end{verbatim}

\begin{Shaded}
\begin{Highlighting}[]
\KeywordTok{ggplot}\NormalTok{(}\DataTypeTok{data =}\NormalTok{ airport_data) }\OperatorTok{+}
\StringTok{  }\KeywordTok{geom_smooth}\NormalTok{(}
    \DataTypeTok{mapping =} \KeywordTok{aes}\NormalTok{(}\DataTypeTok{x =}\NormalTok{ month, }\DataTypeTok{y =}\NormalTok{ mean, }\DataTypeTok{color =}\NormalTok{ origin)}
\NormalTok{  )}
\end{Highlighting}
\end{Shaded}

\begin{verbatim}
## `geom_smooth()` using method = 'loess' and formula 'y ~ x'
\end{verbatim}

\includegraphics{problemset1_files/figure-latex/Airline data question 1-1.pdf}

\begin{Shaded}
\begin{Highlighting}[]
\CommentTok{## The graph shows a clear difference between EWR and the other two airports. It would appear that there is a seasonality for all airports. This we would expect as the summer months see more travelers and probably more families traveling which could account for the increase in delays. Furthermore, we see delays during typically winter months which could be attributed to snowy conditions causing delays. One could use this data to to pick an optimal time to fly and airlines could use this data to scrutinize how flights are scheduled and account for expected delays during specific months.}
\end{Highlighting}
\end{Shaded}

\begin{Shaded}
\begin{Highlighting}[]
\CommentTok{# Which airline experiences more delays and by how much?}

\NormalTok{airline_data <-}\StringTok{ }\NormalTok{flights }\OperatorTok
\StringTok{  }\KeywordTok{group_by}\NormalTok{(carrier) }\OperatorTok
\StringTok{  }\KeywordTok{summarize}\NormalTok{(}\DataTypeTok{mean =} \KeywordTok{mean}\NormalTok{(dep_delay, }\DataTypeTok{na.rm =} \OtherTok{TRUE}\NormalTok{))}
\end{Highlighting}
\end{Shaded}

\begin{verbatim}
## `summarise()` ungrouping output (override with `.groups` argument)
\end{verbatim}

\begin{Shaded}
\begin{Highlighting}[]
\NormalTok{airline_data}
\end{Highlighting}
\end{Shaded}

\begin{verbatim}
## # A tibble: 16 x 2
##    carrier  mean
##    <chr>   <dbl>
##  1 9E      16.7 
##  2 AA       8.59
##  3 AS       5.80
##  4 B6      13.0 
##  5 DL       9.26
##  6 EV      20.0 
##  7 F9      20.2 
##  8 FL      18.7 
##  9 HA       4.90
## 10 MQ      10.6 
## 11 OO      12.6 
## 12 UA      12.1 
## 13 US       3.78
## 14 VX      12.9 
## 15 WN      17.7 
## 16 YV      19.0
\end{verbatim}

\hypertarget{e-challenge-your-results}{%
\subparagraph{(e) Challenge Your
Results}\label{e-challenge-your-results}}

After completing the exploratory analyses from Problem 1d, do you have
any concerns about your findings? How well defined was your original
question? Do you have concerns regarding your answer? Is additional
analysis/different data needed? Comment on any ethical and/or privacy
concerns you have with your analysis.

\emph{I don't believe there is sufficient privacy data of individuals
for there to be any ethical concerns. If there were ticket information
attached to each flight I could see potential privacy issues. I think
the major problem would be assigning more context to some of the
assertions made about the data being used to answer the questions. There
could be many factors that can cause delays of a flight and to rely on
just a few attributes to show which airline or airport does better might
not provide the whole picture. I think the questions I provided are good
starting points but could be refined further or expanded on to allow for
more attributes or variables to provide more insight into what causes
delays.}

\end{document}
